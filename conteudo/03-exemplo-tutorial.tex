%!TeX root=../tese.tex
%("dica" para o editor de texto: este arquivo é parte de um documento maior)
% para saber mais: https://tex.stackexchange.com/q/78101/183146

% Vamos definir alguns comandos auxiliares para facilitar.

% "textbackslash" é muito comprido.
\newcommand{\sla}{\textbackslash}

% Vamos escrever comandos (como "make" ou "itemize") com formatação especial.
\newcommand{\cmd}[1]{\textsf{#1}}

% Idem para packages; aqui estamos usando a mesma formatação de \cmd,
% mas poderíamos escolher outra.
\newcommand{\pkg}[1]{\textsf{#1}}

% A maioria dos comandos LaTeX começa com "\"; vamos criar um
% comando que já coloca essa barra e formata com "\cmd".
\newcommand{\ltxcmd}[1]{\cmd{\sla{}#1}}

\chapter{Do zero ao mínimo com \LaTeX{}}
\label{chap:tutorial}

Preparar um texto para impressão envolve duas coisas:

\begin{description}
\item[Escrever:] digitar, recortar/colar trechos, revisar etc.
\item[Formatar:] definir o tamanho da fonte, o
espaçamento entre parágrafos etc.
\end{description}

Hoje é comum fazer essas duas coisas ao mesmo tempo, graças à visualização
imediata que o computador oferece. No entanto, imagine como era o processo de
produção de um livro nos anos 1970: o autor escrevia seu texto em uma máquina
de escrever e enviava esse material para o editor, que era responsável pela
tarefa de formatá-lo para impressão. O autor muitas vezes inseria anotações
para o editor explicando coisas como ``este parágrafo é uma citação'', e o
editor criava algum mecanismo visual para representar isso.

Não é de se surpreender que, com o surgimento do microcomputador, os primeiros
programas para criação de textos seguissem um funcionamento similar: o autor
digitava e editava seu texto sem formatá-lo visualmente, apenas inserindo
alguns comandos correspondentes a aspectos da formatação que ele depois
revisava na versão impressa. \LaTeX{} é uma ferramenta baseada nesse processo:
você prepara seu texto no editor de sua preferência, insere comandos no texto
que indicam a estrutura do documento e o processa com o \LaTeX{}, que gera um
arquivo \textsc{pdf} formatado. Embora seja um estilo ``antigo'' de trabalhar,
ele é muito eficiente em vários casos. Ou seja, dependendo da situação, pode
ser mais adequado trabalhar fazendo tudo ao mesmo tempo ou dividindo o trabalho
nessas duas fases. De maneira geral:

\begin{itemize}
\item Se você precisa criar páginas diferentes entre si com \emph{layout}
definido manualmente, é melhor usar uma ferramenta que permita trabalhar
visualmente, como LibreOffice Writer, MS-Word, Google Docs etc.;

\item Se você precisa fazer um documento relativamente longo com estrutura
regular (capítulos, seções etc.), é melhor usar ferramentas que formalizam
essa estrutura (como \LaTeX{}) ao invés de ferramentas visuais;

\item Se você precisa fazer um documento envolvendo referências cruzadas,
bibliografia relativamente extensa ou fórmulas matemáticas, é difícil
encontrar outra ferramenta tão eficiente quanto \LaTeX{};

\item Se você precisa criar um documento simples, ambas as abordagens
funcionam bem; cada um escolhe esta ou aquela em função da familiaridade
com as ferramentas;

\item Se você quer que a qualidade tipográfica do resultado seja realmente
excelente, é necessário usar uma ferramenta profissional, como \LaTeX{},
Scribus, Adobe InDesign ou outras; processadores de texto convencionais não
oferecem o mesmo nível de qualidade dessas ferramentas\footnote{A maior
diferença (mas não a única) é o algoritmo que divide cada parágrafo em uma
série de linhas: \TeX{} (desde 1982) e Adobe InDesign (desde 1999) analisam
cada parágrafo como um todo, ao invés de uma linha por vez, para obter
espaçamentos mais homogêneos e menos palavras hifenizadas.}.
\end{itemize}

\section{Visão Geral}

Com \LaTeX{}, você prepara o texto (incluindo as indicações de estrutura) em
um editor de textos qualquer, salva como arquivo de texto puro (``.txt'',
mas é comum usar a extensão ``.tex'' ao invés de ``.txt'') e processa esse
arquivo com o comando ``latexmk'' (``compila'' o documento) para obter o
\textsc{pdf} correspondente. Qualquer editor capaz de salvar arquivos em formato
texto puro, como o bloco de notas do windows, vim, emacs etc. pode ser usado.
Programas como LibreOffice Writer, MS-Word etc. também funcionam, mas
possivelmente vão gerar dores de cabeça porque vão tentar formatar algumas
coisas automaticamente (e de maneira incompatível com \LaTeX{}).

Em geral, é recomendável usar editores projetados especificamente para
trabalhar com \LaTeX{}; eles utilizam cores para distinguir o texto dos
comandos de formatação, automatizam o processo de compilação do documento
e oferecem outras comodidades. O mais usado atualmente é o \TeX{}studio,
que é software livre e funciona em Windows, MacOS e Linux. O editor Visual
Studio Code (\url{code.visualstudio.com}) é voltado para programadores e
tem uma interface às vezes peculiar para outros usuários, mas em conjunto
com a \emph{package} \pkg{LaTeX Workshop} (do editor, não do \LaTeX), é uma
boa opção. O mesmo vale para o editor emacs (\url{www.gnu.org/software/emacs})
e sua package \pkg{AUC\TeX{}}. Ainda outra possibilidade são os editores
\emph{online}; dentre eles, o overleaf (\url{www.overleaf.com}) é o mais usado.

Um documento \LaTeX{} é dividido em duas partes: o \emph{preâmbulo}, onde
você coloca comandos de configuração para o documento, e o \emph{corpo}
do documento em si, que contém o texto propriamente dito. O preâmbulo é
onde você define as características do resultado tipográfico esperado
para o documento como um todo: tipo e tamanho da fonte a usar, posição
dos títulos e subtítulos na página etc. O corpo, por sua vez, consiste no
texto e em alguns comandos indicativos da estrutura.

Dado que configurar o preâmbulo é um tanto complexo e que mesmo no corpo
do texto às vezes há comandos especiais (para a geração da bibliografia
ou tabelas, por exemplo),
usar algum documento existente como base para criar seu texto em geral é
uma boa ideia. O IME/USP oferece um conjunto de modelos adequados para
teses/dissertações, artigos, apresentações e pôsteres (\url{gitlab.com/ccsl-usp/modelo-latex})
que pode ser adaptado para outros usos e outras instituições. Há também uma
família de modelos (\url{www.abntex.net.br}) que procura seguir as normas
da ABNT para diversos tipos de documentos científicos, e algumas publicações
científicas fornecem modelos de acordo com suas diretrizes.

\section{Estrutura de um Documento \LaTeX{}}
\label{sec:basico}

O preâmbulo \LaTeX{} começa com a definição da \emph{classe} a ser utilizada,
que determina boa parte da configuração do documento. As principais classes
são \pkg{book}, \pkg{report} e \pkg{article}; você pode saber mais sobre elas
(e outras) em qualquer texto introdutório sobre \LaTeX{} na Internet (veja a
Seção~\ref{sec:docs}). \pkg{book} e \pkg{report} são as mais adequadas para a
escrita de teses ou dissertações acadêmicas.
A seguir, são carregadas várias \emph{packages} (``\emph{plugins}'') que
acrescentam funcionalidades ou modificam as classes padrão. Qualquer documento
\LaTeX{} utiliza várias delas e é comum que revistas científicas utilizem
packages próprias que pré-definem a formatação esperada para os artigos.
A classe é definida com o comando \ltxcmd{documentclass\{nome-da-classe\}};
packages são carregadas com o comando \ltxcmd{usepackage\{nome-da-package\}}.
Classes e packages podem receber opções adicionais entre colchetes
(\ltxcmd{usepackage[opção1,opção2...]\{nome-da-package\}}); a documentação
de cada package e classe (veja a Seção~\ref{sec:docs}) detalha as opções
disponíveis.

\LaTeX{} ignora quebras de linha e trata sequências de vários espaços como
se fossem apenas um. Isso significa que você pode usar quebras de linha e
espaços no texto que está digitando como ``dicas visuais'' da estrutura do
texto durante a edição. É muito comum fazer isso com listas de itens, por
exemplo. Uma ou mais linhas em branco sinalizam o fim de um parágrafo e o
início de outro. O caractere ``\%'' indica que o restante da linha é um
comentário, ou seja, um trecho de texto que não tem nenhum efeito sobre o
resultado final do documento. Comentários podem ser usados como lembrete sobre
alguma decisão, para indicar um parágrafo que ainda precisa de revisão etc.
Por conta desse significado especial, para inserir um caractere \% ``normal''
no texto é preciso digitar ``\ltxcmd{\%}''.

Como mencionado anteriormente, \LaTeX{} divide o trabalho de produção
de um texto entre a preparação do conteúdo e a definição da forma de
apresentação. Assim, os comandos usados durante a produção do conteúdo
procuram expressar o \emph{significado} de cada elemento, e não sua
aparência. Por exemplo, para realçar uma palavra é comum usar texto
\textit{em itálico}; embora exista um comando especificamente para gerar
textos em itálico em \LaTeX{}, o recomendado é que se utilize o comando
\ltxcmd{emph} (``enfatizado''), pois em alguns casos pode ser melhor
utilizar \textbf{negrito}, \textsc{Versalete} ou outro mecanismo para
dar ênfase a uma palavra. Essa é uma orientação geral para a escrita de
textos com \LaTeX{}: procure definir a estrutura, não a aparência.

%Um exemplo de documento \LaTeX{} simples (lembre-se do significado
%de ``\%''):
Um exemplo de documento \LaTeX{} simples (lembre-se, ``\%'' indica um comentário):

\begin{verbatim}

        % O documento começa com o preâmbulo
        % Vamos usar a classe "book" com fonte no tamanho 11pt
        \documentclass[11pt]{book}
        % Vamos escrever em português do Brasil
        \usepackage[brazilian]{babel}
        % Estas linhas não imprimem nada, apenas definem
        % as informações que serão usadas por "\maketitle"
        \author{Fulano de Tal}
        \title{Começando a usar o \LaTeX{}}
        % Finaliza o preâmbulo e inicia o conteúdo:
        \begin{document}
        % Cria um bloco ou página de título com os dados acima
        \maketitle
        % Capítulos, seções etc. são numerados automaticamente
        \chapter{Cheguei!}
        Oi, Galera!
        % É preciso sinalizar o final do documento
        \end{document}

\end{verbatim}

Esse exemplo mostra como definir o nome de um capítulo. Existem também os
comandos \ltxcmd{section}, \ltxcmd{subsection}, \ltxcmd{subsubsection} e
\ltxcmd{paragraph} (a classe \pkg{book} inclui também \ltxcmd{part}, um nível
acima de \ltxcmd{chapter}). Usar o nome do comando seguido de um asterisco
(\ltxcmd{chapter*} etc.) faz o capítulo/seção não ser numerado nem incluído
no sumário (nem considerado na contagem de capítulos, seções etc.).

\section{Executando \LaTeX{} e Comandos Auxiliares}

Depois de escrever o arquivo \cmd{.tex}, é preciso \emph{compilá-lo}, ou
seja, processá-lo para gerar o \textsc{pdf} desejado. Isso envolve executar,
além do próprio \LaTeX{} (veja a Seção~\ref{sec:versions}), alguns
programas auxiliares (em geral, \cmd{biber} ou \cmd{bibtex} e
\cmd{makeindex}). Nesse processo, \LaTeX{} quase sempre precisa ser
executado três ou mais vezes antes de gerar o \textsc{pdf} final\footnote{A
cada vez, ele gera uma nova versão intermediária do arquivo \textsc{pdf},
mas essas versões têm defeitos, como citações e referências cruzadas
incorretas ou sumário inexistente.}. Por conta dessa complexidade, é comum
utilizar alguma ferramenta para automatizar o processamento. Existem diversas
opções, mas a mais comum é o \cmd{latexmk}, que é capaz de identificar
automaticamente os passos necessários para a geração do documento,
executando os programas na ordem correta quantas vezes forem
necessárias\footnote{É possível personalizar o comportamento de \cmd{latexmk}
com o arquivo de configuração \cmd{latexmkrc}.}. Assim, embora seja possível
gerar o \textsc{pdf} executando apenas \cmd{pdflatex nome-do-arquivo.tex},
acostume-se a compilar o documento sempre com \cmd{latexmk -pdf nome-do-arquivo.tex}.
Note que editores especializados em \LaTeX{} costumam ter uma opção de menu
para a compilação do documento; muitas vezes essa opção simplesmente aciona
\cmd{latexmk}.

\section{Mais sobre Estrutura}

Para criar listas de itens, você pode fazer\footnote{Observe o uso de
espaços no início das linhas com \ltxcmd{item} para deixar a
estrutura visualmente mais clara durante a edição.}:

\begin{verbatim}
        \begin{itemize}
            \item Pedra
            \item Papel
            \item Tesoura
        \end{itemize}
\end{verbatim}

Além de ``itemize'', há também ``enumerate'' (auto-explicativo) e ``description'':

\begin{verbatim}
        \begin{description}
            \item[Pedra:] perde para papel;
            \item[Papel:] perde para tesoura;
            \item[Tesoura:] perde para pedra.
        \end{description}
\end{verbatim}

Citações curtas normalmente são incluídas no fluxo normal do texto e colocadas
entre aspas; para citações mais longas, use \ltxcmd{begin\{quote\}} ou
\ltxcmd{begin\{quotation\}} (este último é mais adequado para citações com
vários parágrafos). Para poesia, use \cmd{verse} (estrofes são separadas por
uma linha em branco e versos são separados por \cmd{\sla\sla{}*}. O asterisco
é opcional; ele instrui \LaTeX{} a manter as linhas na mesma página). A package
\pkg{csquotes} acrescenta recursos sofisticados para citações.

Para inserir uma nota de rodapé, use o comando
\ltxcmd{footnote\{texto da nota\}}\index{Notas de rodapé}. Um espaço
não-separável é indicado pelo caractere til (``\cmd{\textasciitilde{}}'')
e é possível forçar uma quebra de linha com ``\cmd{\sla\sla{}}''. Aspas
tipográficas (``\;'' e `\;') são inseridas com
\`\space\,\`\space\space\,\textquotesingle\,\textquotesingle{} e
\`\space\,\,\textquotesingle. Você pode consultar a lista completa de
símbolos com \textsf{texdoc symbols-a4} ou em \url{www.ctan.org/tex-archive/info/symbols/comprehensive/symbols-a4.pdf}.
Uma outra maneira de encontrar símbolos é usar este sítio: \url{detexify.kirelabs.org/classify.html}.

\section{Figuras e Tabelas (\emph{floats})}
\label{sec:floats}

É possível utilizar \ltxcmd{includegraphics} para acrescentar figuras
ao texto, mas normalmente elas não são inseridas assim. A razão é que,
se você simplesmente inserir uma figura em qualquer lugar, ela pode
ser grande demais para o espaço disponível na página, o que forçará
\LaTeX{} a deixar um espaço em branco e colocá-la na página seguinte.
O mesmo vale para tabelas (criadas com \ltxcmd{begin\{tabular\}}).
Para contornar esse problema, \LaTeX{} possui \emph{floats}, que
são blocos com algum conteúdo cuja localização é flexível: \LaTeX{}
procura colocar um \emph{float} ``perto'' de onde ele foi definido,
mas não necessariamente no lugar exato.

Ao invés de um único comando como ``\ltxcmd{begin\{float\}}'' a
ser usado tanto para figuras quanto para tabelas, \LaTeX{} define
\ltxcmd{begin\{figure\}} e \ltxcmd{begin\{table\}}. Ele faz isso
porque, assim como com capítulos e seções, \LaTeX{} também numera
figuras e tabelas --- mas, para isso, ele precisa saber qual é o tipo de
cada \emph{float}\footnote{É possível criar outros tipos de \emph{float}
também: como pode ser visto no Captítulo~\ref{chap:exemplos}, este
modelo define o tipo \cmd{program}.}. À parte isso, o conteúdo de
um \emph{float} pode ser qualquer coisa mas, em geral, é
\ltxcmd{includegraphics} ou \ltxcmd{begin\{tabular\}}.

Uma consequência importante dos tipos diferentes de \emph{floats} é que
\LaTeX{}\index{Floats!Ordem} garante que a sequência das figuras e a
sequência das tabelas sejam respeitadas (a Figura~6 nunca aparece depois da
Figura~7). No entanto, isso \emph{não} se aplica a \emph{floats} de tipos
diferentes, ou seja, se você definiu a Figura~5, a Tabela~3 e a Figura~6,
elas podem aparecer no documento na ordem ``Figura~5, Tabela~3, Figura~6'',
``Figura~5, Figura~6, Tabela~3'' ou ``Tabela~3, Figura~5, Figura~6''.

\section{Referências Cruzadas}
\label{sec:refs}

É comum que um trecho do texto faça referência a outro trecho (``como
discutimos no Capítulo~X\ldots''). Isso pode ser feito diretamente, mas
se você reorganizar o documento ou acrescentar seções, a numeração pode
mudar. Para evitar esse problema, você pode gerar essas referências
automaticamente com o par de comandos \ltxcmd{label\{nome-sugestivo\}} e
\ltxcmd{ref\{nome-sugestivo\}} (para o número da seção/capítulo) ou
\ltxcmd{pageref\{nome-sugestivo\}} (para o número da página).

Esse mecanismo também é muito útil para figuras e tabelas. Dentro do
\emph{float}, além da figura em si, em geral é uma boa ideia acrescentar
uma legenda com \ltxcmd{caption}\index{Legendas}. Além disso, é possível
inserir um \ltxcmd{label} dentro da legenda para que se possa fazer
referência à figura/tabela no texto (com os comandos \ltxcmd{ref} e
\ltxcmd{pageref})\footnote{Em alguns casos, é possível colocar o
\ltxcmd{label} de uma figura ou tabela fora do comando \ltxcmd{caption}
mas, como em muitos casos isso gera problemas, é um bom hábito sempre
colocá-lo dentro dele.}.


\section{Referências Bibliográficas e Bibliografia}

\enlargethispage{-.5\baselineskip}

A geração de bibliografias no \LaTeX{} é feita através da package
\pkg{biblatex}\index{biblatex} e do programa auxiliar
\cmd{biber}\index{biber}\footnote{Antigamente, usava-se a package
\pkg{natbib}\index{natbib} e o comando auxiliar \cmd{bibtex}\index{bibtex}.
O funcionamento geral dos dois mecanismos é similar e o formato do banco
de dados de ambos é o mesmo.} e envolve três passos:

\begin{enumerate}
\item A criação de um banco de dados, no formato ``.bib'', das obras de
interesse. Esse banco de dados pode incluir obras que não vão ser de fato
referenciadas no documento final. Isso significa que você pode criar um
único banco de dados e utilizá-lo em todos seus documentos\footnote{É
comum criar bancos de dados desse tipo separados por assunto, mas isso
não é necessário.}.

\item A inserção de referências às obras ao longo do texto, usando
diferentes comandos dependendo do caso: \ltxcmd{cite}, \ltxcmd{citet},
\ltxcmd{citep} etc. Como já mencionado, esses comandos estão descritos
na documentação da package \pkg{natbib}\index{natbib} \citep{natbib}.

\item A escolha do estilo bibliográfico (usando as opções da package
\pkg{biblatex}) que formata as citações ao longo do texto e gera a bibliografia
automaticamente através do comando \ltxcmd{printbibliography}.  Normalmente,
apenas as obras efetivamente citadas são incluídas na lista de referências,
mas é possível forçar a inclusão de uma obra sem citá-la explicitamente com
o comando \ltxcmd{nocite}.
\end{enumerate}

O banco de dados é um arquivo de texto contendo uma \emph{entrada} para cada
item da bibliografia e, em cada entrada, uma série de \emph{campos} com os
dados (título, autor etc.). A entrada inclui também uma \emph{chave}, que é
usada para inserir as citações no texto. Há vários tipos de entrada (para
artigos, livros, sítios web etc.) e, para cada tipo, uma lista de campos
possíveis (considere que periódicos normalmente incluem o número do volume,
mas teses não). O exemplo abaixo é um livro cuja chave é ``dissertjourney'';
ele pode ser citado com o comando \ltxcmd{cite\{dissertjourney\}}:

\begin{verbatim}
        @book{dissertjourney,
            author    = {Carol M. Roberts},
            title     = {The Dissertation Journey},
            publisher = {Corwin},
            year      = 2010,
            edition   = 2,
            location  = {Thousand Oaks, CA},
        }
\end{verbatim}

Em alguns casos, \LaTeX{} troca as letras maiúsculas definidas em
\ltxcmd{title} para minúsculas. Para evitar que isso afete siglas
ou nomes próprios, basta colocá-los entre chaves (``Automated
Application-Level Checkpointing of \{MPI\} Programs'').

% Esta informação é muito pouco relevante...
%Observe que existem dois formatos comumente usados para escrever títulos
%de artigos, livros etc:
%
%\begin{description}
%  \item[Title case:] Substantivos, adjetivos e verbos (além de nomes
%  próprios e siglas) são escritos com a primeira letra maiúscula (``Um
%  Exemplo de Título no Estilo Title Case''). Em geral, a regra não se
%  aplica ao título de artigos ou capítulos de livro, apenas aos livros
%  dos quais eles fazem parte;
%
%  \item[Sentence case:] O título é escrito como qualquer outra frase
%  (``Um título só tem maiúsculas em abreviaturas, como ABNT, ou nomes
%  próprios'').
%\end{description}
%
%\enlargethispage{-.5\baselineskip}
%
%Cada estilo de bibliografia utiliza um desses formatos e, portanto, é
%desejável que o banco de dados funcione corretamente com ambos. No
%entanto, nem sempre é claro quais palavras devem ser iniciadas com letra
%maiúscula ao usar \emph{title case} e, por conta disso, não há um sistema
%automático em \LaTeX{} para adaptar títulos a ele. Sendo assim, como fazer
%um banco de dados bibliográfico capaz de funcionar com os dois formatos?
%A solução é sempre inserir os títulos dos itens no banco de dados seguindo
%o formato \emph{title case}. Se o estilo utiliza esse formato, o título
%é reproduzido na bibliografia como digitado no banco de dados. Se o estilo
%usa \emph{sentence case}, o texto (exceto a primeira letra) é convertido
%para letras minúsculas. Para evitar que isso afete siglas e nomes próprios,
%basta colocá-los entre chaves (``Automated Application-Level Checkpointing
%of \{MPI\} Programs'').

Os campos \textsf{author} e \textsf{publisher} podem incluir uma lista
de nomes separados por \textsf{and}; biblatex reconhece que cada nome é
composto por nome e sobrenome, às vezes com partículas como ``de'', ``dos''
ou ``von'' e, dependendo do estilo bibliográfico, pode abreviar nomes, mudar
sobrenomes para caixa alta etc. Isso evidentemente não funciona quando o autor
é, na verdade, uma instituição; nesses casos, basta colocar o nome inteiro da
instituição entre chaves (``\{Universidade de São Paulo --- Sistema Integrado
de Bibliotecas\}'') para que biblatex não faça alterações desse tipo. Se o
nome é longo, pode ser interessante definir o campo \textsf{shortauthor}.

A fonte mais detalhada de informações sobre o banco de dados é a documentação
da package \pkg{biblatex} \citep[em especial as seções 2.1.1 e 2.2.2]{biblatex},
mas o material ali é um tanto denso.
Há muito material introdutório ao formato ``.bib'' e ao bibtex disponível
\emph{online}, e você pode se inspirar em exemplos para criar seu banco de
dados bibliográfico. Além disso, ferramentas como Zotero\index{Zotero} ou
Mendeley\index{Mendeley} (o uso de uma delas é altamente recomendado!)
podem exportar para o formato .bib. Observe que \pkg{biblatex}
\index{biblatex} oferece recursos bastante sofisticados para o tratamento de
referências e bibliografias. Se você precisar de alguma funcionalidade
especial, consulte a documentação do pacote ou a Internet; é quase certeza
que \pkg{biblatex} oferece uma solução.


\section{Fórmulas Matemáticas}

\enlargethispage{-.5\baselineskip}

A diagramação de fórmulas matemáticas tem regras específicas; assim, para
criar fórmulas em \LaTeX{}, é preciso usar um comando para iniciar o modo
matemático. Isso pode ser feito de duas formas:

\begin{itemize}
  \item Pequenas fórmulas no meio do texto ($E=mc^2$) são inseridas com
  \cmd{\$\emph{fórmula}\$} (e, portanto, para inserir um caractere \$
  normal no texto, é preciso usar \cmd{\sla{}\$}).

  \item Fórmulas mais longas ou que devem aparecer em um parágrafo
  separado são inseridas com \cmd{\sla{}[\emph{fórmula}\sla{}]} (ou
  \ltxcmd{begin\{displaymath\}}).
\end{itemize}

No modo matemático, letras são interpretadas como variáveis e espaços
em branco são ignorados (\LaTeX{} usa o contexto da fórmula para
definir o espaçamento). Para inserir um espaço explicitamente, use
\ltxcmd{quad} ou \ltxcmd{enspace}. Para inserir texto ``normal'' em
uma fórmula matemática, use \ltxcmd{text\{texto\}} (para texto de fato)
ou \ltxcmd{mathit\{texto\}} (para nomes de variáveis ou funções com
mais de uma letra). Pode ser necessário deixar um espaço no início do
texto para evitar que ele fique colado com o caractere matemático que
o antecede.

Usando \ltxcmd{begin\{equation\}}, a fórmula recebe um número (que
aparece à direita) ao qual você pode se referir no texto usando os
comandos ``\ltxcmd{ref}'' e ``\ltxcmd{eqref}'' (``\textsf{conforme
vimos na equação \ltxcmd{ref\{eq:bhaskara\}\ldots}}'' ou
``\textsf{de acordo com \ltxcmd{eqref\{eq:bhaskara\}\ldots}}'').
\ltxcmd{begin\{equation*\}} (incluindo o *) elimina o número e é,
portanto, equivalente a \ltxcmd{begin\{displaymath\}}. Há outros
comandos similares, como \cmd{align}, \cmd{multline} e \cmd{gather},
definidos e documentados na package \pkg{amsmath}, e todos têm
a variante com ``*''.

\section{Formatação Manual}

Às vezes é preciso inserir formatação de forma manual; os comandos mais
importantes são:
\ltxcmd{emph} (texto \emph{enfatizado}, em geral itálico),
\ltxcmd{texttt} (texto \texttt{teletype}, imitando um
terminal de texto ou uma impressora),
\ltxcmd{textit} (\textit{itálico}),
\ltxcmd{textbf} (\textbf{negrito}),
\ltxcmd{textsf} (fonte \textsf{sem serifa}),
\ltxcmd{textsc} (texto \textsc{Versalete} --- nem todas
as fontes oferecem essa possibilidade),
\ltxcmd{normalsize} (tamanho normal),
\ltxcmd{small} (tamanho reduzido),
\ltxcmd{footnotesize} (ainda menor),
\ltxcmd{scriptsize} (ainda menor),
\ltxcmd{tiny} (ainda menor),
\ltxcmd{large} (tamanho aumentado),
\ltxcmd{Large} (ainda maior),
\ltxcmd{LARGE} (ainda maior),
\ltxcmd{Huge} (ainda maior),
\ltxcmd{vspace\{\sla{}baselineskip\}} (deixa uma linha em branco),
\ltxcmd{begin\{center\}} (centraliza parágrafos),
\ltxcmd{begin\{flushleft\}} (alinha parágrafos à esquerda),
\ltxcmd{begin\{flushright\}} (alinha parágrafos à direita)\footnote{É
altamente recomendável carregar a package \pkg{ragged2e} (já incluída
neste modelo) e utilizar \ltxcmd{Center}, \ltxcmd{FlushLeft} e
\ltxcmd{FlushRight} ao invés de \ltxcmd{center}, \ltxcmd{flushleft}
e \ltxcmd{flushright}.},
\ltxcmd{babelhyphenation} (permite ``ensinar'' \LaTeX{} como hifenizar
uma lista de palavras, veja \cmd{texdoc babel}; note que, em geral, a
hifenização automática de \LaTeX{} é excelente),
\ltxcmd{-} (sugere uma possível hifenização localizada),
\ltxcmd{linebreak}[0--4] (sugere uma quebra de linha; o número indica
quão forte é a sugestão, ou seja, 4 faz a quebra obrigatória; se o
parágrafo é justificado, a linha quebrada também é justificada),
\ltxcmd{newline} ou \cmd{\sla\sla} (força uma quebra de linha; a
linha \emph{não} é justificada nesse caso),
\ltxcmd{pagebreak}[0--4] (sugere uma quebra de página; como
\ltxcmd{linebreak}, o número indica quão forte é a sugestão; o texto
da página é espalhado verticalmente de maneira a fazer a última linha
alinhada com o final das demais páginas) e
\ltxcmd{newpage} (força uma quebra de página; o final da página
\emph{não} é alinhado com o final das demais páginas nesse caso).

Mas, como discutido na Seção~\ref{sec:basico}, não é recomendável
usar esses comandos ao longo do texto: o ideal em \LaTeX{} é expressar
o significado de cada elemento, não a sua forma de apresentação,
pois isso permite que você faça alterações na formatação com mais
facilidade. Assim, quando os recursos pré-definidos do \LaTeX{}
(\ltxcmd{itemize}, \ltxcmd{chapter} etc.) não forem suficientes,
o mais adequado é definir comandos novos, em geral usando os comandos
de formatação mencionados acima. Esse é um tópico avançado, mas você
pode consultar o início do arquivo \LaTeX{} deste capítulo para alguns
exemplos simples.

\section{Formatos de Imagem}
\label{sec:graficos}

Podemos classificar imagens em quatro categorias:

\begin{enumerate}
    \item Imagens fotográficas ou escaneadas, que consistem em um conjunto
    de \emph{pixels} coloridos sem organização previsível.

    \item Ilustrações, que consistem em curvas e figuras geométricas
    que formam uma imagem completa, como um objeto ou uma paisagem.
    Apesar de lidarem com abstrações geométricas ao invés de meros
    \emph{pixels}, elas ainda são desenhadas de forma totalmente manual
    em programas como Inkscape ou CorelDraw!.

    \item Diagramas, que são ilustrações estruturadas, como fluxogramas,
    grafos ou diagramas UML, criadas com ferramentas como Draw.io,
    LibreOffice Draw ou Microsoft Visio. Graças à sua estrutura intrínseca,
    os programas podem automatizar, ao menos parcialmente, o trabalho de
    posicionar e alinhar cada elemento.

    \item Gráficos de dados, como gráficos de pizza ou de barras. A
    geração desses gráficos, em geral, é quase totalmente automatizada
    por ferramentas como Gnuplot, R, LibreOffice Calc ou Microsoft Excel.
\end{enumerate}

Em \LaTeX{}, é possível importar imagens fotográficas nos formatos
\textsc{png} e \textsc{jpeg} e imagens dos demais tipos no formato
\textsc{pdf}. Além disso, \LaTeX{} tem recursos para criar ilustrações,
diagramas e gráficos diretamente, mas usá-los em geral não é trivial.
Em particular, a package \pkg{tikz} oferece bons recursos para a
criação de ilustrações e diagramas (incluindo funções pré-prontas
para formas geométricas, grafos, matrizes etc.) e é fácil usá-la
para traçar linhas ou curvas simples. Você também pode criar gráficos
de dados ou de funções matemáticas com a package \pkg{pgfplots}.
\cmd{Gnuplot}, com o \emph{driver} \cmd{lua tikz}\footnote{
\url{www.gnuplot.info/docs\_5.2/Gnuplot\_5.2.pdf\#section*.516}}, e
\cmd{matplotlib}, com o \emph{backend} \textsc{pgf}\footnote{
\url{matplotlib.org/users/pgf.html}}, são capazes de exportar gráficos
de dados na forma de comandos para \pkg{tikz} (garantindo maior
consistência visual entre o texto principal e o gráfico), e o programa
\cmd{Asymptote} tem excelente integração com \LaTeX{}.

\section{Detalhes da Linguagem}

Há quatro estilos típicos de comandos \LaTeX{}:

\begin{itemize}
\item Comandos que se referem a um parâmetro; por exemplo,
\ltxcmd{emph\{um texto\}} significa ``escreva a frase
`um texto' com ênfase'' (em geral, itálico). As chaves delimitam o início
e o final do escopo sobre o qual o comando tem efeito. Aqui entram também
comandos como \ltxcmd{title} e \ltxcmd{author},
que não escrevem nada diretamente mas definem o título e autoria do documento
(essa informação é usada, por exemplo, por \ltxcmd{maketitle}).

\item Comandos que se referem a um parâmetro que é um bloco grande de
texto, possivelmente vários parágrafos; por exemplo, \ltxcmd{begin\{center\}}
um texto \ltxcmd{end\{center\}} faz ``um texto'' (que podem ser vários
parágrafos) ser centralizado.

\item Comandos que ativam alguma opção; por exemplo, \ltxcmd{itshape}
significa ``ative o modo itálico''. Nesse caso, o texto vai ser impresso
em itálico até outro comando selecionar outro estilo de fonte. Se o comando
for inserido dentro de um bloco delimitado por chaves, ele ``perde o
efeito'' após o caractere de fecha-chaves (exemplo: ``\{\ltxcmd{itshape\{\}}
Fulano de Tal\} é meu nome'' será impresso como ``\textit{Fulano de Tal} é
meu nome''). Você normalmente não vai utilizar esse estilo de comando, mas
ele é útil em alguns casos.

\item Comandos que fazem o programa escrever algo específico; por exemplo,
em várias classes padrão o comando \ltxcmd{maketitle} gera
uma página de título com o nome do trabalho, autor etc.
\end{itemize}

Nos dois últimos, não é preciso usar chaves após o comando. Ainda assim, as
chaves podem ser colocadas e muitas vezes isso é bom: sem elas, \LaTeX{}
entende que o caractere espaço que se segue a esses comandos serve apenas
como separador em relação ao que vem a seguir. Por conta disso, ele ignora
esse espaço. Quando isso não é o que se deseja, a solução é usar as chaves:
\ltxcmd{itshape\{\}}.
Vale observar que alguns comandos aceitam mais de um parâmetro, às vezes
entre chaves, às vezes entre colchetes. Você pode descobrir a sintaxe
correta para cada caso lendo a documentação de cada comando.

\section{Versões do \LaTeX{}}
\label{sec:versions}

Assim como há packages para o \LaTeX{}, o próprio \LaTeX{} é, na verdade, um
conjunto de extensões para o programa \TeX{}. Assim, se você encontrar
referências a ``\TeX{}'' ou a ``plain \TeX{}'', basta saber que esse é o
sistema que funciona ``por baixo'' do \LaTeX{}.

\LaTeX{} é um sistema em evolução (desde os anos 80!). Uma das consequências
disso é que há, na verdade, quatro versões diferentes dele:

\begin{enumerate}
\item \LaTeX{} ``tradicional'', que gera arquivos em formato \textsc{dvi}
que, por sua vez, precisam ser convertidos para o formato \textsc{pdf}.
Essa versão não é capaz de usar as fontes instaladas no sistema; ela só
pode usar fontes adaptadas para uso com o \LaTeX{}. Hoje em dia não há
boas razões para usar essa versão.

\item pdf\LaTeX{}, que gera arquivos \textsc{pdf} e dá suporte a alguns
recursos avançados de tipografia adicionais. É a versão mais usada hoje
em dia, embora também só possa usar as fontes adaptadas para uso com o
\LaTeX{}.

\item \XeLaTeX{} que, além dos recursos do pdf\LaTeX{}, opera internamente
em UTF-8 (ou seja, funciona melhor com múltiplas línguas) e pode funcionar
não só com as fontes adaptadas para o \LaTeX{} como também com as fontes
instaladas no sistema. \XeLaTeX{} foi muito importante ao ser lançado,
mas atualmente a comunidade está mais empenhada em evoluir o sistema com
\LuaLaTeX{}.

\item \LuaLaTeX{}, que oferece os mesmos recursos que o \XeLaTeX{} e
também pode ser estendido internamente com mais facilidade (através da
linguagem de programação Lua).
\end{enumerate}

Todas essas versões são instaladas quando você instala \LaTeX{} na
sua máquina. Em geral, se você pretende escrever apenas com línguas no
alfabeto latino e não pretende usar fontes diferentes das disponíveis
por padrão no \LaTeX{}, qualquer das três versões modernas (pdf\LaTeX{},
\XeLaTeX{} e \LuaLaTeX{}) é adequada, mas pdf\LaTeX{} é um pouco mais
rápido. Se você pretende usar outros alfabetos, gostaria de escolher
fontes diferentes ou precisa de recursos tipográficos avançados
(\cmd{texdoc fontspec}, \cmd{texdoc unicode-math}), use \LuaLaTeX{}.

\section{Limitações do \LaTeX{} e Algumas Dicas}
\label{sec:limitations}

Como qualquer ferramenta, \LaTeX{} tem limitações e características
indesejáveis:

\begin{itemize}
    % \linebreak[0]{} -> sugestão (não-obrigatória) de quebra de linha
    \item A linguagem é muito prolixa: é bastante tedioso escrever
    coisas como ``\ltxcmd{begin\linebreak[0]{}\{itemize\}}'' etc.
    Linguagens como asciidoc (\url{asciidoctor.org}), markdown
    (\url{commonmark.org}), bookdown (\url{bookdown.org}) e
    reStructuredText (\url{sphinx-doc.org}) operam de maneira similar
    a \LaTeX{}, mas sua sintaxe é bem mais enxuta. Elas funcionam
    muito bem para a geração de páginas web, mas \LaTeX{} oferece
    mais recursos e geralmente produz resultados impressos melhores.

    \item \LaTeX{} gera muitas mensagens pouco importantes durante
    o processamento do documento, o que dificulta a identificação
    de problemas. Além disso, quando ocorrem erros durante esse
    processamento, as mensagens explicativas de \LaTeX{} muitas vezes
    são confusas ou, pior, não indicam o problema real que causou a falha.

    \item \LaTeX{} procura ser uma linguagem \emph{declarativa}, ou seja,
    os comandos buscam expressar o que se deseja e não como fazer algo
    (``este texto é um título'' e não ``pule duas linhas, selecione uma
    fonte maior, escreva este texto, pule mais duas linhas e selecione a
    fonte de tamanho padrão''). No entanto, ela é insuficiente em algumas
    situações, obrigando o usuário a utilizar vários comandos, às vezes
    obscuros, para obter resultados relativamente simples.

    \item Há diversas packages para personalizar os aspectos básicos
    da formatação final do documento, como o tipo de fonte, tamanho dos
    títulos das seções, espaçamento etc. No entanto, quando se quer
    fazer modificações maiores, é preciso lidar com partes complexas da
    linguagem e diversos comportamentos surpreendentes.

    \item Às vezes há incompatibilidades entre packages; em alguns casos,
    isso pode ser contornado mudando a ordem em que elas são carregadas,
    mas em outros pode simplesmente não ser possível combiná-las.

    \item A colocação automática dos \emph{floats} em geral funciona bem, mas
    nem sempre. Isso acontece porque \LaTeX{} decide o posicionamento de cada
    \emph{float} individualmente, sem levar em conta os próximos \emph{floats},
    e nunca reavalia essa decisão. No exemplo da Seção~\ref{sec:floats}, se a
    ordem ``Figura~5, Tabela~3, Figura~6'' for aceitável, esse vai ser o
    resultado, mesmo que a ordem ``Tabela~3, Figura~5, Figura~6'' seja melhor.
    Apenas se não for possível encontrar um lugar aceitável para a Figura~5
    imediatamente (ou seja, na página atual) é que \LaTeX{} processa os
    \emph{floats} seguintes e, depois, procura novamente um lugar para ela.
    Por isso, depois que seu trabalho estiver finalizado, vale a pena
    avaliar se a colocação dos \emph{floats} pode ser melhorada; se sim,
    mudar o lugar em que eles são definidos no documento (veja algumas
    dicas em \cite{floats2014}) pode fazer \LaTeX{}
    gerar um resultado melhor (mas lembre-se que isso só faz sentido depois
    que o documento estiver pronto, pois qualquer mudança no texto pode
    mudar totalmente a posição final dos \emph{floats}).

    \item O algoritmo que \LaTeX{} usa para quebrar páginas funciona
    bem, minimizando linhas órfãs ou viúvas e garantindo uma distribuição
    homogênea do texto na página, mas não é excelente. Assim, se
    houver quebras de página ruins no seu texto final, pode ser útil
    modificá-las manualmente. Uma técnica usada por editores profissionais
    é mudar ligeiramente a altura do texto impresso em algumas páginas,
    melhorando a distribuição geral do texto. Para isso, ao invés
    de comandos como \ltxcmd{pagebreak} ou \ltxcmd{newpage}, o mais
    adequado é usar \ltxcmd{enlargethispage\{\sla{}baselineskip\}} (ou
    \cmd{-1\sla{}baselineskip}). Esse comando instrui \LaTeX{} a fazer
    a página ligeiramente maior (ou menor), tornando possível acomodar
    mais uma linha de texto (ou uma linha a menos). Em documentos frente
    e verso, lembre-se de sempre garantir que a página adjacente também
    tenha seu tamanho modificado para que a alteração não seja tão
    perceptível. Um outro truque às vezes útil é aplicar o comando
    \ltxcmd{looseness=1} (ou -1) a um parágrafo, que faz \LaTeX{} tentar
    reorganizar as quebras de linha de maneira a fazer o parágrafo ter
    uma linha a mais (ou a menos), se isso for possível.

    % LaTeX atualmente usa utf-8 por padrão; esta informação já
    % era pouco útil antes, agora é quase totalmente irrelevante
    %\item Como muitos outros sistemas de texto, \LaTeX{} pode usar mais de
    %um padrão para a codificação de caracteres acentuados (através da
    %configuração da package \pkg{inputenc}). Alguns anos atrás,
    %o mais comum era o ISO-8859-1, também conhecido como latin1 (esse é o
    %nome usado no \LaTeX) ou Windows-1252; atualmente, o mais comum é o
    %UTF-8 (default em versões recentes de \LaTeX). De maneira geral, é
    %simples reconhecer e resolver os problemas
    %causados por inconsistências na codificação (seja trocando a opção
    %de \pkg{inputenc}, seja recodificando o arquivo), mas arquivos ``.bib''
    %são um caso especial: biblatex (usado neste modelo) funciona normalmente
    %com caracteres acentuados, mas bibtex oficialmente não tem suporte a eles
    %(embora em geral funcione corretamente). Além disso, é bastante comum que
    %arquivos desse tipo sejam compartilhados por várias pessoas, com diferentes
    %configurações. Para evitar problemas com os acentos
    %nesse caso, uma possibilidade é representar os caracteres acentuados
    %usando comandos \LaTeX{}: \cmd{\sla\textquotesingle{}a} para á,
    %\cmd{\sla{}c\{c\}} para cedilha etc., independentemente da
    %codificação usada no texto\footnote{Você pode consultar os comandos
    %desse tipo mais comuns em \url{en.wikibooks.org/wiki/LaTeX/Special_Characters}.
    %Observe que a dica sobre o pingo do i \emph{não} é mais
    %válida atualmente; basta usar \cmd{\sla\textquotesingle{}i}.}.

    \item As classes padrão (\pkg{book}, \pkg{article} etc.) não foram criadas
    para serem facilmente modificadas, o que deu origem a inúmeras packages
    voltadas para possibilitar a personalização de diversos aspectos da
    apresentação final do documento. Esse mecanismo não é ideal, por diversas
    razões. Por conta disso, existe um conjunto de versões alternativas dessas
    classes (\pkg{scrbook} no lugar de \pkg{book}, \pkg{scrartcl} no lugar de
    \pkg{article} etc.) chamado \pkg{KOMA-Script}, com mais recursos e mais
    possibilidades de customização. A classe \pkg{memoir} tem o mesmo objetivo,
    mas procura dar suporte a livros e artigos com uma única classe. Ambas
    abordagens são muito boas, mas a maioria dos modelos usados por revistas e
    outras publicações é baseada nas classes padrão. A versão 3 de \LaTeX{}
    está em desenvolvimento com vistas a resolver boa parte dos problemas
    atuais do sistema, mas ainda deve demorar muitos anos para ficar pronta.
    \ConTeXt{} é um ``irmão mais novo'' de \LaTeX{} com diversas
    vantagens, mas com sintaxe diferente e que ainda não é tão popular.
\end{itemize}
