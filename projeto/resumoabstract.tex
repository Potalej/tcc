%!TeX root=../tese.tex
%("dica" para o editor de texto: este arquivo é parte de um documento maior)
% para saber mais: https://tex.stackexchange.com/q/78101/183146

% As palavras-chave são obrigatórias, em português e em inglês, e devem ser
% definidas antes do resumo/abstract. Acrescente quantas forem necessárias.
% \palavrachave{Problema de N-corpos}
% \palavrachave{Simulação numérica}
% \palavrachave{Integração numérica}

% \keyword{N-body problem}
% \keyword{Numerical simulation}
% \keyword{Numerical integration}

% O resumo é obrigatório, em português e inglês. Estes comandos também
% geram automaticamente a referência para o próprio documento, conforme
% as normas sugeridas da USP.
\abstract{
A evolução de sistemas gravitacionais pode ser descrita através de Problema de N-corpos (PNC), cuja simulação numérica permite visualizar e testar hipóteses e propriedades qualitativas do Problema, como a respeito de sua evolução e setas do tempo, conforme a Dinâmica de Formas, o caos e sua estabilidade e sua complexificação em modelos mais robustos, como com colisões elásticas. Diversos métodos numéricos já foram desenvolvidos para simulações do PNC em grande escala mas os erros de precisão costumam ser irrisórios para as simulações galáticas. Nesse sentido, estudamos a simulação numérica de pequeno porte tendo em vista sua otimização e precisão nos resultados, principalmente com o uso de métodos simpléticos e da mecânica hamiltoniana, para visualizar corretamente o qualitativo esperado e investigar as questões do caos, que permanecem como um problema em aberto no PNC, e a aplicação de modelos mais robustos.
}