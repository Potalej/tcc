%!TeX root=../tese.tex
%("dica" para o editor de texto: este arquivo é parte de um documento maior)
% para saber mais: https://tex.stackexchange.com/q/78101/183146

%% ------------------------------------------------------------------------- %%
\section{Introdução}
\label{sec:introducao}

A evolução de sistemas gravitacionais é complexa e as soluções de suas equações de movimento até então encontradas convergem lentamente, sendo inutilizáveis na prática. Nesse sentido, simular numericamente esse tipo de sistema permite visualizar aproximações das soluções e verificar hipóteses e proposições acerca de sua evolução.

Para isso, utilizaremos o Problema de N-Corpos (PNC) como \textit{toy-model} de sistemas gravitacionais. Estudaremos algumas de suas propriedades qualitativas e o seu uso na descrição quantizada da \textit{Dinâmica de Formas}, que é um modelo de gravidade relacionalista emergente baseado em Relatividade Geral. 

Numericamente, estudaremos a simulação do PNC passando pela determinação de valores iniciais, os métodos de integração numérica comuns e os simpléticos, formas de lidar com as singularidades e também as dificuldades da implementação prática das simulações utilizando as linguagens de programação Python e Fortran.

Nossos objetivos serão visualizar as \textit{setas do tempo} propostas pela Dinâmica de Formas, investigar a caracterização do caos dentro do PNC e avaliar sua estabilidade, e a possibilidade de estender os resultados qualitativos e numéricos do PNC para sistemas de gravidade mais complexos, partindo do exemplo do PNC com colisões perfeitamente elásticas.