%!TeX root=../tese.tex
%("dica" para o editor de texto: este arquivo é parte de um documento maior)
% para saber mais: https://tex.stackexchange.com/q/78101/183146

%% ------------------------------------------------------------------------- %%
\chapter{Introdução}
\label{cap:introducao}

\textbf{
  A introdução precisa ter um CONTEXTO, um OBJETIVO e uma HIPÓTESE.
}

Desde o início da formalização da Mecânica por Isaac Newton, diversos sistemas mecânicos
receberam maior ou menor atenção a depender de sua relevância. Um dos mais famosos é o
chamado \textit{Problema de N-corpos} (PNC). O problema de N-corpos consiste na suposição de $N$
partículas dispostas no vácuo e interagindo a distância através da força gravitacional ou
da eletromagnética. A dificuldade - e a denominação \textit{problema} - aparece quando $N$ é um
número suficientemente grande, pois resolver as equações de movimento analiticamente passa
a ser, no geral, impossível. De fato, o sistema passa a ser um problema já em $N = 3$.

Alguns resultados qualitativos a respeito de PNCs com valores iniciais específicos foram obtidos e
descritos nas obras de Lagrange, Painlevé e Sundman, em especial acerca de colisões entre
partículas e a existência de soluções. Além disso, algumas teorias de gravidade baseadas em
Relatividade Geral, como a Dinâmica de Formas (\textit{Shape Dynamics}), possuem simplificações
descritas pelo PNC e que trazem resultados gerais interessantes, como o modelo de 
Barbour-Bertotti (BB) e sua redução dimensional do Problema.

Na aplicação geral do problema, como no lançamento de foguetes e sondas e no estudo de sistemas
planetários simples, o uso de Análise Numérica para a integração das equações de movimento
do PNC é fundamental. A infinitude de métodos existentes faz ser necessário o estudo de
quais tipos de métodos numéricos se adequam melhor ao objetivo da simulação, que nem sempre
vão no caminho da precisão.

Simulações de grande escala, como o AbacusSummit, visam mais o panorama geral do que o local,
ao contrário de simulações de rotas de sondas que passem por estilingues gravitacionais, como
as Voyagers. Assim, enquanto as simulações do primeiro tipo podem se utilizar de métodos
numéricos com menor precisão mas maior rapidez, as do segundo tipo necessitam de métodos mais
precisos e consequentemente mais custosos.

De forma geral, há uma categoria de métodos voltados para a integração numérica de sistemas
mecânicos fechados, chamados \textit{Métodos Simpléticos} (MS). Tais métodos baseiam-se na
estrutura simplética de sistemas mecânicos, como o PNC, e utilizam da conservação de alguns
valores, como a energia total, para fornecer uma maior estabilidade da solução e a garantia
de verossimilhança para simulações com um longo período de tempo. Isso é particularmente
interessante em simulações planetárias, nas quais as órbitas são, no geral, estáveis e 
quase periódicas.

Para todo tipo de propósito, no entanto, é um desafio encontrar os métodos e ferramentas mais
adequados, pois para cada método existe uma variedade de programas e linguagens de programação
que se adequam em maior ou menor medida.

O objetivo deste trabalho, assim, é estudar a simulação numérica do PNC e a possibilidade de
utilizá-la na busca de resultados qualitativos do Problema. Estudaremos a formalização e alguns
resultados interessantes já descobertos do PNC, classes de métodos numéricos e sua adequação
ao Problema, correções numéricas a posteriori e a possibilidade de estender os resultados obtidos
para PNCs mais elaborados, como com colisões elásticas.

Ao final, pretendemos mostrar que os métodos simpléticos são os mais adequados para simulações
de propósito geral, e que a extensão dos resultados é possível se a complexificação mantiver
a estrutura simplética do PNC.