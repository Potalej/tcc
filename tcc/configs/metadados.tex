
% O arquivo com os dados bibliográficos para biblatex; você pode usar
% este comando mais de uma vez para acrescentar múltiplos arquivos
\addbibresource{tcc/bibliografia/bibliografia.bib} % gerais
\addbibresource{tcc/bibliografia/numerico.bib}     % metodos numericos
\addbibresource{tcc/bibliografia/ncorpos.bib}      % n-corpos
\addbibresource{tcc/bibliografia/mecanica.bib}     % mecanica
\addbibresource{tcc/bibliografia/sd.bib}           % dinamica de formas

% Este comando permite acrescentar itens à lista de referências sem incluir
% uma referência de fato no texto (pode ser usado em qualquer lugar do texto)
%\nocite{bronevetsky02,schmidt03:MSc, FSF:GNU-GPL, CORBA:spec, MenaChalco08}
% Com este comando, todos os itens do arquivo .bib são incluídos na lista
% de referências
%\nocite{*}

% É possível definir como determinadas palavras podem (ou não) ser
% hifenizadas; no entanto, a hifenização automática geralmente funciona bem
\babelhyphenation{documentclass latexmk soft-ware} % todas as línguas
\babelhyphenation[brazilian]{Fu-la-no}
\babelhyphenation[english]{what-ever}

% Estes comandos definem o título e autoria do trabalho e devem sempre ser
% definidos, pois além de serem utilizados para criar a capa, também são
% armazenados nos metadados do PDF.
\title{
    % Obrigatório nas duas línguas
    titlept={Simulação numérica do problema de N-corpos gravitacional},
    titleen={Numerical simulation of gravitational N-body problem},
    % Opcional, mas se houver deve existir nas duas línguas
    subtitlept={},
    subtitleen={},
}

\author[masc]{Octavio Augusto Potalej}

% Para TCCs, este comando define o supervisor
\orientador[masc]{Profº. Dr. Eduardo Colli}

% Se não houver, remova; se houver mais de um, basta
% repetir o comando quantas vezes forem necessárias
% \coorientador{Prof. Dr. Ciclano de Tal}
% \coorientador[fem]{Profª. Drª. Beltrana de Tal}

% A página de rosto da versão para depósito (ou seja, a versão final
% antes da defesa) deve ser diferente da página de rosto da versão
% definitiva (ou seja, a versão final após a incorporação das sugestões
% da banca).
\defesa{
  nivel=tcc, % mestrado, doutorado ou tcc
  % É a versão para defesa ou a versão definitiva?
  definitiva,
  % É qualificação?
  %quali,
  programa={Matemática Aplicada e Computacional},
  membrobanca={Prof. Dr. Eduardo Colli (orientador) -- IME-USP},
  % Em inglês, não há o "ª"
  %membrobanca{Prof. Dr. Fulana de Tal (advisor) -- IME-USP [sem ponto final]},
  membrobanca={Prof. Dr. Clodoaldo Grotta Ragazzo -- IME-USP},
  membrobanca={Prof. Dr. Claudio Hirofume Asano  -- IME-USP},
  % Se não houve bolsa, remova
  %
  % Norma sobre agradecimento por auxílios da FAPESP:
  % https://fapesp.br/11789/referencia-ao-apoio-da-fapesp-em-todas-as-formas-de-divulgacao
  %
  % Norma sobre agradecimento por auxílios da CAPES (Portaria 206,
  % de 4 de Setembro de 2018):
  % https://www.in.gov.br/materia/-/asset_publisher/Kujrw0TZC2Mb/content/id/39729251/do1-2018-09-05-portaria-n-206-de-4-de-setembro-de-2018-39729135
  %
  %apoio={O presente trabalho foi realizado com apoio da Coordenação
  %       de Aperfeiçoamento\\ de Pessoal de Nível Superior -- Brasil
  %       (CAPES) -- Código de Financiamento 001}, % o código é sempre 001
  %
  %apoio={This study was financed in part by the Coordenação de
  %       Aperfeiçoamento\\ de Pessoal de Nível Superior -- Brasil
  %       (CAPES) -- Finance Code 001}, % o código é sempre 001
  %
  %apoio={Durante o desenvolvimento deste trabalho, o autor recebeu\\
  %       auxílio financeiro da FAPESP -- processo nº aaaa/nnnnn-d},
  %
  %apoio={During the development if this work, the author received\\
  %       financial support from FAPESP -- grant \#aaaa/nnnnn-d},
  %
  % apoio={Durante o desenvolvimento deste trabalho o autor
  %        recebeu auxílio financeiro da XXXX},
  local={São Paulo},
  data=2024-12-13, % YYYY-MM-DD
  % A licença do seu trabalho. Use CC-BY, CC-BY-NC, CC-BY-ND, CC-BY-SA,
  % CC-BY-NC-SA ou CC-BY-NC-ND para escolher a licença Creative Commons
  % correspondente (o sistema insere automaticamente o texto da licença).
  % Se quiser estabelecer regras diferentes para o uso de seu trabalho,
  % converse com seu orientador e coloque o texto da licença aqui, mas
  % observe que apenas TCCs sob alguma licença Creative Commons serão
  % acrescentados ao BDTA. Se você tem alguma intenção de publicar o
  % trabalho comercialmente no futuro, sugerimos a licença CC-BY-NC-ND.
  direitos={CC-BY}, % Creative Commons Attribution 4.0 International License
  %direitos={Autorizo a reprodução e divulgação total ou parcial
  %          deste trabalho, por qualquer meio convencional ou
  %          eletrônico, para fins de estudo e pesquisa, desde que
  %          citada a fonte.},
  % Isto deve ser preparado em conjunto com o bibliotecário
  fichacatalografica={% Este codigo depende das packages calc, setspace e ragged2e
% (ja incluidas no modelo LaTeX do IME-USP)
\begingroup\centering\singlespacing\small
\hyphenrules{nohyphenation}\hbadness=10000
Ficha catalográfica elaborada com dados inseridos pelo(a) autor(a)\\
Biblioteca Carlos Benjamin de Lyra\\
Instituto de Matemática e Estatística\\
Universidade de São Paulo\par
\vspace{2\baselineskip}\hrule\vspace{.8\baselineskip}
\RaggedRightRightskip 0pt plus 30pt minus 0pt\relax
\RaggedRightParfillskip 20pt plus 40pt minus 10pt\relax
\ttfamily\hspace{2em}\begin{minipage}[t]{125mm}
\RaggedRight\sloppy\setlength{\parindent}{\widthof{123}}

\noindent Potalej, Octavio Augusto

Simulação numérica do problema de N-corpos gravitacional / Octavio Augusto Potalej; orientador, Eduardo Colli. - São Paulo, 2024.

125 f.: il.

\vspace{1\baselineskip}


\vspace{1\baselineskip}


Trabalho de Conclusão de Curso (Graduação) - Matemática Aplicada / Instituto de Matemática e Estatística / Universidade de São Paulo.

Bibliografia



\vspace{1\baselineskip}


\vspace{1\baselineskip}


1. PROBLEMAS DE N-CORPOS. 2. EQUAÇÕES DIFERENCIAIS ORDINÁRIAS. 3. ANÁLISE NUMÉRICA APLICADA. 4. MECÂNICA HAMILTONIANA. I. Colli, Eduardo. II. Título.

\end{minipage}\par
\vspace{1\baselineskip}\hrule\vspace{.5\baselineskip}\rmfamily
Bibliotecárias do Serviço de Informação e Biblioteca\\
Carlos Benjamin de Lyra do IME-USP, responsáveis pela\\
estrutura de catalogação da publicação de acordo com a AACR2:\\
Maria Lúcia Ribeiro CRB-8/2766; Stela do Nascimento Madruga CRB 8/7534.
\par\endgroup},
}
