% Este codigo depende das packages calc, setspace e ragged2e
% (ja incluidas no modelo LaTeX do IME-USP)
\begingroup\centering\singlespacing\small
\hyphenrules{nohyphenation}\hbadness=10000
Ficha catalográfica elaborada com dados inseridos pelo(a) autor(a)\\
Biblioteca Carlos Benjamin de Lyra\\
Instituto de Matemática e Estatística\\
Universidade de São Paulo\par
\vspace{2\baselineskip}\hrule\vspace{.8\baselineskip}
\RaggedRightRightskip 0pt plus 30pt minus 0pt\relax
\RaggedRightParfillskip 20pt plus 40pt minus 10pt\relax
\ttfamily\hspace{2em}\begin{minipage}[t]{125mm}
\RaggedRight\sloppy\setlength{\parindent}{\widthof{123}}

\noindent Potalej, Octavio Augusto

Simulação numérica do problema de N-corpos gravitacional / Octavio Augusto Potalej; orientador, Eduardo Colli. - São Paulo, 2024.

125 f.: il.

\vspace{1\baselineskip}


\vspace{1\baselineskip}


Trabalho de Conclusão de Curso (Graduação) - Matemática Aplicada / Instituto de Matemática e Estatística / Universidade de São Paulo.

Bibliografia



\vspace{1\baselineskip}


\vspace{1\baselineskip}


1. PROBLEMAS DE N-CORPOS. 2. EQUAÇÕES DIFERENCIAIS ORDINÁRIAS. 3. ANÁLISE NUMÉRICA APLICADA. 4. MECÂNICA HAMILTONIANA. I. Colli, Eduardo. II. Título.

\end{minipage}\par
\vspace{1\baselineskip}\hrule\vspace{.5\baselineskip}\rmfamily
Bibliotecárias do Serviço de Informação e Biblioteca\\
Carlos Benjamin de Lyra do IME-USP, responsáveis pela\\
estrutura de catalogação da publicação de acordo com a AACR2:\\
Maria Lúcia Ribeiro CRB-8/2766; Stela do Nascimento Madruga CRB 8/7534.
\par\endgroup