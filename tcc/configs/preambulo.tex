% A opção twoside (frente-e-verso) significa que a aparência das páginas pares
% e ímpares pode ser diferente. Por exemplo, as margens podem ser diferentes ou
% os números de página podem aparecer à direita ou à esquerda alternadamente.
% Mas nada impede que você crie um documento "só frente" e, ao imprimir, faça
% a impressão frente-e-verso.
%
% Aqui também definimos a língua padrão do documento
% (a última da lista) e línguas adicionais.
%\documentclass[12pt,twoside,brazilian,english]{book}
\documentclass[12pt,twoside,english,brazilian]{book}

% Ao invés de definir o tamanho das margens, vamos definir os tamanhos do
% texto, do cabeçalho e do rodapé, e deixamos a package geometry calcular
% o tamanho das margens em função do tamanho do papel. Assim, obtemos o
% mesmo resultado impresso, mas com margens diferentes, se o tamanho do
% papel for diferente.
\usepackage[a4paper]{geometry}

\geometry{
  textwidth=152mm,
  hmarginratio=12:17, % 24:34 -> com papel A4, 24mm + 152mm + 34mm = 210mm
  textheight=237mm,
  vmarginratio=8:7, % 32:28 -> com papel A4, 32mm + 237mm + 28mm = 297mm
  headsep=11mm, % distância entre a base do cabeçalho e o texto
  headheight=21mm, % qualquer medida grande o suficiente, p.ex., top - headsep
  footskip=10mm,
  marginpar=20mm,
  marginparsep=5mm,
}

% Vários pacotes e opções de configuração genéricos; para personalizar o
% resultado, modifique estes arquivos.
\usepackage{helper/helper}
\input{extras/basics}
\input{extras/languages}
\input{extras/fonts}
\input{extras/floats}
\input{extras/imeusp-thesis} % capa, páginas preliminares e alguns detalhes
\input{extras/imeusp-formatting}
% O comando backmatter desabilita a numeração de capítulos.
\backmatter

\pagestyle{backmatter}

% Espaço adicional no sumário antes das referências / índice remissivo
\addtocontents{toc}{\vspace{2\baselineskip plus .5\baselineskip minus .5\baselineskip}}

% A bibliografia é obrigatória

\printbibliography[
  title=\refname\label{bibliografia}, % "Referências", recomendado pela ABNT
  %title=\bibname\label{bibliografia}, % "Bibliografia"
  heading=bibintoc, % Inclui a bibliografia no sumário
]

% ÍNDICE REMISSIVO
% \printindex % imprime o índice remissivo no documento (opcional)

\input{extras/bibconfig}
\input{extras/hyperlinks}
%\nocolorlinks % para impressão em P&B
\input{extras/source-code}
\input{extras/utils}

% Diretórios onde estão as figuras; com isso, não é preciso colocar o caminho
% completo em \includegraphics (e nem a extensão).
\graphicspath{{figuras/},{logos/}}

% Comandos rápidos para mudar de língua:
% \en -> muda para o inglês
% \br -> muda para o português
% \texten{blah} -> o texto "blah" é em inglês
% \textbr{blah} -> o texto "blah" é em português
\babeltags{br = brazilian, en = english}

% Bibliografia
\usepackage[
  style=extras/plainnat-ime, % variante de autor-data, similar a plainnat
  %style=alphabetic, % similar a alpha
  %style=numeric, % comum em artigos
  %style=authoryear-comp, % autor-data "padrão" do biblatex
  % style=apa, % variante de autor-data, muito usado
  % style=abnt,
]{biblatex}

% \let\oldthebibliography\thebibliography
% \let\endoldthebibliography\endthebibliography
% \renewenvironment{thebibliography}[1]{
%   \begin{oldthebibliography}{#1}
%     \setlength{\itemsep}{2em}
%     \setlength{\parskip}{2em}
% }
% {
%   \end{oldthebibliography}
% }
