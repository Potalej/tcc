\begin{table}[H]
    \begin{tabular}{S[table-format=1.8, round-precision=8]|S[table-format=1.8, round-precision=8]}
    \multicolumn{1}{c}{$\vet x$} & \multicolumn{1}{c}{$\vet y$} \\
    \hline
        -1.11290666 & -0.10354093 \\
         0.0        &  0.76488086 \\
         1.11290666 & -0.66133993 \\
    \hline
    \end{tabular}
    \quad
    \begin{tabular}{S[table-format=1.9, round-precision=9]|S[table-format=1.9, round-precision=9]}
    \multicolumn{1}{c}{$\vet p_x$} & \multicolumn{1}{c}{$\vet p_y$} \\
    \hline
         0.53645589 &  1.06454049 \\
        -0.96477633 & -1.16156992 \\
         0.42832044 &  0.09702943 \\
    \hline
    \end{tabular}
    \caption{Posições e velocidades iniciais para o problema-modelo \ref{probmodelo:3corpos_energia_positiva}.}
    \label{tab:3corpos_energiapositiva}
\end{table}