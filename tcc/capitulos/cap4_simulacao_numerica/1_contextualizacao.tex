% * CONTEXTUALIZACAO E OBJETIVOS
%
%   Este sera um capitulo de aplicação. Falar brevemente de como serao
%   feitos os testes e mais abertamente o que se quer aqui.
%

O objetivo deste último capítulo é a aplicação da teoria apresentada. Começando pela implementação computacional, discutimos a escolha da linguagem de programação Fortran e a estrutura do programa final. Passamos então para as questões de valores iniciais, apresentando o método utilizado para condicionar as integrais primeiras.

A questão das colisões vem em seguida, pois existem outras formas computacionais de lidar com singularidades e aproximações intensas no PNCG além da inclusão de colisões perfeitamente elásticas.

Algumas questões sobre o corretor numérico apresentado na seção \ref{secao:corretor_numerico} também precisam ser discutidas, como o seu uso e seu custo de computação associado. Além disso, este envolve a resolução de um sistema de equações (para mais de uma integral primeira em consideração), então foi necessário escolher um método de resolução adequado.

Por fim, apresentamos algumas simulações de muitos corpos e com condições iniciais específicas para testar os resultados da Dinâmica de Formas enunciados na seção \ref{secao:dinamica_de_formas}.

Todas as simulações e testes realizados neste trabalho foram aplicados em um computador de mesa (não dedicado) com processador \textit{Intel Core i5 CPU 760 @ 2.80GHz} e com 16 GB de RAM.

% Como já explicitado, neste trabalho são consideradas os sistemas puramente gravitacionais e conservativos, além de pontuais, ou seja, cada corpo é um ponto individual. As simulações de grande porte utilizadas na astronomia e na cosmologia no geral consideram, no lugar de pontos de massa, funções de densidade $\rho$ e um campo gravitacional $\nabla \phi$, de modo que o potencial fica dado por uma equação de Poisson
% \begin{equation*}
%     \Delta \phi = - 4 \pi G \rho.
% \end{equation*}
% O campo é então discretizado através de métodos numéricos voltados para equações diferenciais parciais, e neste ponto é feita a integração numérica temporal. Isto é necessário por uma série de razões práticas, como a implementação de gases, galáxias e afins, indo bem além dos objetivos deste trabalho. [TALVEZ COLOCAR ISSO NA INTRODUÇÃO DO TRABALHO, E NÃO AQUI]

% Sendo assim, existe neste trabalho a necessidade de calcular direta e rapidamente expressões não-lineares com ordem quadrática, como o potencial gravitacional, sem o uso de aproximações como o algoritmo de Barnes-Hut [TALVEZ FALAR MELHOR SOBRE TAMBÉM, SÓ PRA NÃO FICAR JOGADO]. Isso levou à escolha do Fortran como linguagem de programação para as simulações, e suas particularidades serão melhor exploradas adiante.