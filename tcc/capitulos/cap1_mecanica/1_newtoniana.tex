\subsection{Equações de movimento na mecânica newtoniana}
Na mecânica newtoniana, os objetos são descritos matematicamente a partir de dois referenciais absolutos: o espaço e o tempo. As equações de movimento que parametrizam tais objetos são dadas pela 2ª Lei de Newton.

\begin{definition}[2ª Lei de Newton]
    Um objeto de massa $m > 0$, posição inicial $\vet q_0 \in M \subseteq \R^n$ e sujeito a um campo de forças $\vet F: I \times M \to M$, $I \subseteq \R$, tem sua trajetória $\vet q: I \subseteq \R \to M$ descrita pelo seguinte Problema de Cauchy:
    \begin{equation}\label{eq:segunda_lei_de_newton}
        \begin{cases}
            m \ddvet q (t) = \vet F(t, \vet q(t)), \forall t \in I, \\
            \vet q(t_0) = \vet q_0, \dvet q(t_0) = \dvet q_0, \ t_0 \in I.
        \end{cases}
    \end{equation}
\end{definition}

No caso do PNCG, tratado com detalhes no capítulo \ref{capitulo:pncg}, o campo de forças depende somente da posição, e como as equações são autônomas (isto é, não dependem explicitamente do tempo), escrevemos na forma de notação reduzida:
\begin{equation}
    m \ddvet q = F(\vet q),
\end{equation}
ou, ainda, tomando $\vet v(t) \in T_{\vet q(t)} M$, onde $T_{\vet q(t)} M$ é o espaço tangente a $M$ em $\vet q(t)$\footnote{Os conceitos de espaços e fibrados tangente e cotangente são melhor explorados na seção \ref{subsecao:k-formas}. No momento, para visualizar $TM$ qualquer basta pensar em uma derivada temporal.}, e definindo $\vet v_0 = \vet v(t_0)$, podemos escrever na seguinte forma:
\begin{equation}
    \begin{cases}
        \dvet q = \vet v, \\
        \dvet v = \frac{1}{m} \vet F(\vet q), \\
        \vet q(t_0) = \vet q_0, \vet v(t_0) = \vet v_0.
    \end{cases}
\end{equation}

Nos sistemas newtonianos, um conceito que aparece naturalmente é a energia total.

\begin{definition}[Função potencial]
    Quando $\vet F$ é um campo gradiente, isto é, existe uma função suave $V:M \to \R$ tal que $\vet F = - \nabla V$, dizemos que $V$ é uma \textbf{função potencial} do sistema.
\end{definition}

\begin{definition}[Energia total]
    A energia total $E$ de um sistema newtoniano com função potencial $V$ é dada por
    \begin{equation}\label{eq:energia_total}
        E(t, \vet q, \vet v) = \dfrac{1}{2} m \norma{\vet v}^2 + V(\vet q),
    \end{equation}
    onde $\frac{1}{2} m \norma{\vet v}^2$ é chamada de \textbf{energia cinética} e a função potencial é chamada de \textbf{energia potencial}.
\end{definition}

A energia total, quando não depende explicitamente do tempo e o campo é fechado (ou seja, não há forçantes), possui a propriedade de ser uma \textit{integral primeira} do sistema, isto é, é constante em relação ao tempo.

\begin{definition}[Integral primeira]\label{def:integral_primeira}
    Um observável $f$ (uma função que depende de $t$, $\vet q$ e $\vet v$) é dito uma \textbf{integral primeira} se for pelo menos $\continuo^1$ e
    \begin{equation}
        \der{}{t} f (t, \vet q(t), \vet v(t)) = 0, \forall t \in I.
    \end{equation}
\end{definition}

\begin{theorem}\label{teorema:energia_total}
    A energia total $E(t, \vet q(t), \vet v(t))$ de um sistema newtoniano é uma integral primeira.
\end{theorem}
\begin{Proof}
    Basta derivar:
    \begin{align*}
        \der{}{t} E(t, \vet q(t), \vet v(t))
        &= \dfrac{1}{2} m \der{}{t}(\norma{\vet v}^2) + \der{}{t} V(\vet q(t))  \\
        &= \dfrac{1}{2} m 2 \prodint{\vet v}{\dvet v} + \prodint{\nabla V}{\vet v} \\
        &= \prodint{\vet v}{\vet F} - \prodint{\vet F}{\vet v} = 0.
    \end{align*}
\end{Proof}

Isso significa que a partir de um par de valores iniciais $(\vet q_0, \vet v_0)$, obtém-se a energia total de toda a trajetória. Analiticamente, quando um sistema em $\R^n$ possui $k$-integrais primeiras não-triviais, é possível remodelar o sistema para trabalhar em um espaço reduzido $\R^{n-k}$ \citep[93-94]{de_queiroz_barros_mecanica_1995}. Já numericamente, isso é interessante do ponto de vista da precisão numérica, pois, uma vez que a energia deve se manter constante durante a trajetória, sua variação numérica é uma forma de mensurar os erros nas aproximações.