\section{Integrais primeiras}
O PNCG tridimensional possui dez integrais primeiras, cada uma correspondente a uma simetria dentro do sistema.

\begin{theorem}
    O PNCG possui como integrais primeiras não-triviais a energia total $E$ (correspondente à invariância temporal), momento angular total $\vet J$ (correspondente à invariância sob rotações), o momento linear total $\vet P$ e a trajetória $\vet G(t)$ do centro de massas $\vet q_{cm}$ (correspondentes à invariância sob translações).
\end{theorem}
\begin{Proof}
    Para a energia total, a demonstração já foi feita no Teorema \ref{teorema:energia_total}.

    Para o momento linear total $\vet P = \sum_{a=1}^N \vet p_a$, basta aplicar a definição \ref{def:integral_primeira}:
    \begin{equation*}
        \der{\vet P}{t} = \sum_{a=1}^{N} \der{\vet p_a}{t} = \sum_{a=1}^N \vet F_a = \vet 0,
    \end{equation*}
    pois o sistema é fechado.

    Para a trajetória do centro de massas, temos:
    \begin{equation}
        \vet G(t) = M \vet q_{cm} - t \vet P
        \Rightarrow
        \der{G}{t} = M \dfrac{\vet P}{M} - \vet P = \vet 0.
    \end{equation}
    No caso em que $\vet P = \vet 0$, a conservação de $G(t)$ equivale à conservação de $\vet q_{cm}$.

    Por fim, o momento angular total:
    \begin{equation}
        \der{\vet J}{t} 
        = \sum_{a=1}^N \vet q_a \times \dvet p_a
        = \sum_{a=1}^N \sum_{b\neq a}^{N} \dfrac{G m_a m_b}{r_{ab}^3} \vet q_a \times \vet q_b = \vet 0.
    \end{equation}
\end{Proof}

O fato de o centro de massas do Problema ter uma rota linear e seu momento generalizado ($\vet P$) ser uma integral primeira significa que se o centro de massas partir da origem e o momento linear total for nulo, o sistema não deverá se mover. Isso é particularmente interessante para simulações numéricas, mas mesmo analiticamente ainda é uma propriedade que facilita o estudo.